\documentclass[11pt,a4paper]{ivoa}
\input tthdefs

\newcommand{\xtype}[1]{\texttt{#1}}
\newcommand{\ucd}[1]{\texttt{#1}}

\usepackage{listings}
\lstloadlanguages{XML,sh}
\lstset{flexiblecolumns=true,basicstyle=\small,tagstyle=\ttfamily}
\usepackage[utf8]{inputenc}
\usepackage{todonotes}
\usepackage{enumerate}
\usepackage[section]{placeins}

\title{IVOA Architecture}

\ivoagroup{Technical Coordination Group}

\author{Patrick Dowler}
\author{Janet Evans}
\author{Christophe Arviset}
\author{Severin Gaudet}
\author{IVOA Technical Coordination Group}

\editor{Patrick Dowler, Janet Evans}

\previousversion{IVOAArchitecture-1.0-20101123}

\begin{document}

\begin{abstract}
This note describes the technical architecture of the International Virtual Observatory Alliance 
(IVOA). The description is decomposed 
into three levels. Level 0 is a general, high level summary of the IVOA Architecture. 
Level 1 provides more details about components and functionalities, still without 
being overly technical. Finally, Level 2 displays how the IVOA standards fit into 
the IVOA Architecture. This architecture enables the community of resource providers to
implement the FAIR principles: Findable, Accessible, Interoperable, and Reusable.
\end{abstract}

\section*{Acknowledgments}

\section{IVOA Architecture Level 0}

\begin{figure}[h]
\centering
\includegraphics[width=0.9\textwidth]{archdiag0.pdf}
\caption{IVOA Architecture Level 0}
\label{fig:architecture0}
\end{figure}

Astronomy produces large amounts of data of many kinds, coming from various sources: 
science  space missions, ground based telescopes, theoretical models, compilation of 
results, etc.  These data are usually managed by large data centres or smaller teams
and they provide  the scientific community with data and/or computing services 
through the Internet. This is the Resource Layer. 

The ``consumers'' of these data and computing services, be it individual researchers, 
research teams or computer systems, interact with the User Layer. 

The Virtual Observatory (VO) is the necessary ``middle layer'' framework connecting the 
Resource Layer to the User Layer in a seamless and transparent manner. Like the web 
which enables end  users and machines to access documents and services 
wherever and however they are stored, the VO enables the astronomical scientific
community to access astronomical resources wherever and however they are stored by 
the astronomical data and services providers. The VO provides a technical framework 
for the providers to enable users to discover data collections and services 
(``Findable`) and to use them for science and public outreach (``Accessible''). To 
enable these functionalities in perpetuity, it defines some core astronomically-oriented 
standards so data from different providers can be combined (``Interoperable'') to enable
new scientific discoveries (``Reusable'').

The IVOA Architecture uses terms ``Finding'', ``Getting'', ``Using'', and ``Sharing``;
these are collectively equivalent to ``Findable'', ``Accessible'', ``Interoperable'', and
``Reusable'': the FAIR principles have always been the basis of the IVOA Architecture
even before the term was formally coined \citep{Wilkinson2016}. The world wide
astronomy community has long supported sharing and reusability of data (e.g. through standards 
like FITS \citep{std:FITS}). Within the IVOA community, interoperability has been the cornerstone
of development of standards and the concepts of reusability and interoperability go beyond 
metadata and data as they also guide the development of standards for applications, services, 
and infrastructure for research, education, and public outreach.

\section{IVOA Architecture Level 1}

Level 1 of the IVOA architecture is an extension to the Level 0, displaying more 
details about the functionalities and building blocks within the different layers. 
For completeness, part of the description is repeated  from  the  Level 0, so the  
Level 1 description can be used as a self-contained block.

\begin{figure}[h]
\centering
\includegraphics[width=0.9\textwidth]{archdiag1.pdf}
\caption{IVOA Architecture Level 1}
\label{fig:architecture1}
\end{figure}

Astronomy produces large amounts of data of many kinds, coming from various sources:  
science  space  missions, ground  based telescopes, theoretical models, compilation  
of results, etc. These  data are usually managed by large data centres or 
smaller teams. These providers provide the scientific community with data and
computing services through the Internet. These resources provided can be:

\begin{enumerate}
\item data collections (images, spectra, time series, theoretical  models, catalogues, etc.)  
with their associated descriptive metadata and access services.  
\item storage services for users and for processing 
\item computing services to process data from data collections and from users
\end{enumerate}

This is the Resource Layer. 

The ``consumers''  of  these data and computing services, be it individual researchers,
research teams or computer systems, interact with the User Layer of the IVOA architecture.
These interactions can be through browser based applications in a typical web browser,
standalone desktop applications or scriptable applications that can be used in interactive 
and batch mode by a computer. 

The Virtual Observatory is the necessary ``middle layer'' framework that connects the
Resource Layer to the User Layer in a seamless and transparent manner. Like the web
that enables end users and machines to access transparently documents and services
wherever and however they are stored, the VO enables the astronomical scientific community
to access astronomical resources wherever and however they are stored by the astronomical
data and services providers. The VO provides a technical framework for the providers and
the consumers to share their data and services (``Sharing''). Registries function as the ``yellow
pages'' of the VO, collecting metadata about data resources and information services into a 
queryable database. Like the VO resources and services themselves, the registry is also 
distributed. Replicas exist around the network, both for redundancy and for more specialized 
collections. Access to data and metadata collections is available through Data Access Protocols, 
which specify a uniform way of getting data and metadata from various different providers. 
To allow these functionalities, the definition of some core astronomically-oriented standards 
(``VO Core'') is necessary. In particular, defining common formats and data  models and using
common semantics is required to have a uniform and common description of astronomical datasets 
so they can become interoperable and queryable through standard query languages to enable cross
analysis amongst various datasets. Additional standards are required within the User Layer to enable
user authentication to proprietary datasets and storage elements as well as interoperability amongst 
VO applications (``Using'').

\section{IVOA Architecture Level 2}

Level 2 of the IVOA Architecture is similar to the Level 1, but adds all the IVOA 
standards in  their corresponding layer. Some  standards  have  already  been  approved 
and recommended (blue boxes with an outer line) while others are still a work in progress
(blue boxes without an outer line).

\begin{figure}[h]
\centering
\includegraphics[width=0.9\textwidth]{archdiag2.pdf}
\caption{IVOA Architecture Level 2}
\label{fig:architecture2}
\end{figure}

Note that this list  (and standard status) will naturally evolve with time. Driven by
science use cases and implementation experience, existing standards will be updated and
new standards will be identified and added to that Fig.~\ref{fig:architecture2}. 

The following sections of this document provide a summary description of each current standard, 
including a description of where it fits in the overall IVOA architecture along with links with 
other IVOA standards. The links 
(arrows) in the diagrams below indicate a ``used by'' dependency: readers wanting to 
understand the full scope of a standard will also need to review the other standards 
represented in each box of the diagrams. 
The standards shown in blue boxes are the subject of the section; 
boxes in light gray can be found in a different section/diagram. 
In some cases, green boxes are used to show external standards that provide an 
important or notable part of the IVOA standard. 
A white rounded box around several standards indicates a group of standards that share 
dependencies (to simplify the diagrams).

\section{Authentication and Authorization}

Authentication is a process by which you verify that someone is who they claim they 
are. Authorization is the process of establishing if the user (who is already authenticated), 
is permitted to have access to a resource. The authentication and authorization architecture 
is primarily an endorsement of existing
industry standards and technologies that suit the use cases of the IVOA community. The standards
in this area provide some recommendations and ``glue'' so that participants (application
developers, metadata and data providers, and resource providers) can easily implement interoperable
systems.

Authentication and authorization are generally orthogonal to other standards and there are minimal
direct dependencies on them. Implementors of other standards (e.g. Sec.~\ref{dal} and \ref{gws}) 
``combine'' these A \& A recommendations where necessary to support local policies and requirements.

\begin{figure}[h]
\centering
\includegraphics[width=0.28\textwidth]{ivoa-arch-auth.pdf}
\caption{Authentication and Authorization Standards and Dependencies}
\label{fig:authdeps}
\end{figure}

\subsection{SSO}

The Single-Sign-On (SSO) \citep{2017ivoa.spec.0524T} profile describes authentication mechanisms. Approved client-server 
authentication mechanisms are described for the IVOA single-sign-on profile: No Authentication; 
HTTP Basic Authentication; TLS with passwords; TLS with client certificates; Cookies; Open 
Authentication; Security Assertion Markup Language; OpenID. Normative rules are given for 
the implementation of these mechanisms, mainly by reference to pre-existing standards.

\subsection{CDP}

The Credential Delegation Protocol (CDP) \citep{2010ivoa.spec.0218P} allows a client program to delegate a user's credentials 
to a service such that that service may make requests of other services in the name of that 
user. The protocol defines a REST service that works alongside other IVO services to enable 
such a delegation in a secure manner. In addition to defining the specifics of the service 
protocol, the standard document describes how a delegation service is registered in an IVOA registry 
along with the services it supports. The specification also explains how one can determine 
from a service registration that it requires the use of a supporting delegation service. 

\subsection{GMS - Draft}

The Group Membership Service (GMS, WD) specification describes a service interface for determining 
whether a user is a member of a group. Membership information can be used to protect access to 
proprietary resources. When an authorization decision is needed (whether to grant or deny access 
to a proprietary resource), a call to GMS can be made to see if the requesting user is a member 
of the group assigned to protect the resource in question. Examples of proprietary resources are 
wide ranging but include: observation data and metadata and scarce or limited services and 
infrastructure. Because this specification details how a single group can protect multiple, 
potentially distributed, resources, it allows for the representation of teams with common 
authorization rights. The members of such teams can span multiple organizations but can be 
managed within a single service. In this way, GMS offers an interoperable, flexible, and 
scalable mechanism for sharing proprietary assets with a potentially dynamic set of team members. 

\section{Application and Format Standards}

Application and Format Standards are focused on standards that support data formats and 
protocols that enable astronomy software tools to interoperate and communicate. IVOA 
members have recognised that building a monolithic tool that attempts to fulfil all the 
requirements of all users is impractical, and it is a better use of our limited resources 
to enable individual tools to work together better. One element of this is defining 
common file formats for the exchange of data between different applications. Another 
important component is a messaging system that enables the applications to share data 
and take advantage of each other's functionality.

\begin{figure}[h]
\centering
\includegraphics[width=0.48\textwidth]{ivoa-arch-apps.pdf}
\caption{Application and Format Standards and Dependencies}
\label{fig:appsdeps}
\end{figure}

\subsection{HiPS}

Hierarchical Progressive Survey (HiPS) \citep{2017ivoa.spec.0519F} is a hierarchical scheme for the description, storage and 
access of sky survey data. The system is based on hierarchical tiling of sky regions at 
finer and finer spatial resolution which facilitates a progressive view of a survey, and 
supports multi-resolution zooming and panning. HiPS uses the HEALPix tessellation of the 
sky as the basis for the scheme and is implemented as a simple file structure with a direct 
indexing scheme that leads to practical implementations. 

\subsection{MOC}

The Multi-Order Coverage Map (MOC) \citep{2019ivoa.spec.1007F} is a method to specify coverage as an arbitrary sky regions. 
The goal is to be able to provide a very fast comparison mechanism between coverage maps. 
The mechanism is based on the HEALPix sky tessellation algorithm. It is essentially a 
simple way to map regions of the sky into hierarchically grouped predefined cells. 

\subsection{VOTable} 

The VOTable \citep{2019ivoa.spec.1021O} format is an XML standard for the interchange of data represented as a set of tables. 
In this context, a table is an unordered set of rows, each of a uniform structure, as specified 
in the table description (the table metadata). Each row in a table is a sequence of table cells, 
and each of these contains either a primitive data type, or an array of such primitives. VOTable 
is derived from the Astrores format \citep{astrores}, itself modeled on the FITS Table format 
\citep{std:FITS}; VOTable was designed to be close to the FITS Binary Table format. 

\subsection{SAMP}

The Simple Application Messaging Protocol (SAMP) \citep{2009ivoa.spec.0421B} is a messaging protocol that enables astronomy 
software tools to interoperate and communicate. SAMP supports communication between 
applications on the desktop and in web browsers, and is also intended to form a framework 
for more general messaging requirements. 

\section{Semantics Standards}

An interoperable data infrastructure needs common languages in many
places: From common designations of units to labels for physical
quantities, from common names of reference frames and time scales to
mutually understandable subject categories, from relationship types
between VO resources (``this service publishes images from A and spectra
fom B'') to fixed names for the messengers that produced the signals
recorded.

The VO's semantics standards provide the basis of forming such
consensual ``vocabularies'', which are, at their root, sets of labeled
concepts (which in turn are sets of entities clients deal with).  We
also take care that, whereever possible, our vocabularies are
interoperable with the rest of the semantic web by adopting the W3C's
Resource Description Framework RDF.

The vocabularies themselves are usually introduced by standards that use
them and are then maintained on the VO's repository of
vocabularies\footnote{\url{http://www.ivoa.net/rdf}}.  In some cases,
however, we go beyond RDF, usually because the labels have an intrinsic
syntax.  In these cases, the Semantics WG issues separate standards
defining how to build and interpret these labels.  Currently, this is
the case for unit strings and for the Unified Content Descriptors
discussed below.

\begin{figure}[h]
\centering
\includegraphics[width=0.32\textwidth]{ivoa-arch-semantics.pdf}
\caption{Semantics Standards and Dependencies}
\label{fig:semdeps}
\end{figure}

\subsection{Vocabularies}

Vocabularies in the VO is a Recommendation for how to build and use
consensus vocabularies in the Virtual Observatory.  It supports both
``soft'' vocabularies based on the Simple Knowledge Organisation System
SKOS and ``hard'' vocabularies based on RDF schema, where the latter
organise their concepts in strict trees.  The ``hard'' vocabularies
enable simple inference with relatively little effort on the side of
the clients.  An example could be ``give me all links giving auxiliary
data for the current dataset'' in datalink, where vocabulary-aware
clients will also return links tagged as weight maps, errors, or noise
estimates.

This standard also details the maintenance of the VO's vocabulary
repository, in particular as regards adding vocabularies or concepts
within them.

\subsection{VOUnits}

VOUnits \citep{2014ivoa.spec.0523D} describes how to serialise unit strings within the Virtual
Observatory, in particular (but by no means limited to) in the
\xmlel{unit} attribute in VOTable.  It hence defines the atomic units,
prefixes applicable, and the syntax of expressions using such prefixed
atomic units. 

An important design goal was consistency with other standards (BIPM,
ISO/IEC and the IAU) that are relevant in the astronomical community.
The intention is that units written to conform to VOUnits will likely
also be parsable by other well-known parsers.

\subsection{UCD}

Unified Content Descriptors (UCD) citep{2019ivoa.spec.1007G} are a way to denote astronomical data
quantities.  The UCD formalism first defines a list of ``atoms'', in
effect a controlled vocabulary with a hierarchy implied through dots
(e.g., \ucd{pos} denotes positions, \ucd{pos.eq} equatorial positions);
this list is currently maintained as an Endorsed Note.

The atoms can then be combined into more complex labels containing
qualifications.  For instance, \ucd{phot.mag;em.opt.V} denotes a
magnitude in the V band, \ucd{phot.flux;em.opt.V} a flux in the same
band.  The UCD standard defines how these compound UCDs are built, and
the UCD list defines restrictions as to where in complete UCDs atoms can
be used: some atoms can only be ``primary'', others are only available
as qualifiers.  For instance, \ucd{stat.error} can only appear at the
start of a UCD, which ensures that ``Error in redshift'' will be encoded
as \ucd{stat.error;src.redshift} rather than the other way round.

The UCD ecosystem is completed by another standard on how new atoms are
adopted to the list of UCDs.

\section{Registry Standards}

The IVOA Registry provides a mechanism with which VO applications can discover and select 
resources that are relevant for a particular scientific problem. The VO specification defines 
the operation of this system. It is based on a general, distributed model composed of searchable 
and publishing registries. There are three components: (a) an interface for harvesting publishing 
registries, which builds upon the Open Archives Initiative Protocol for Metadata Harvesting. 
(b) A VOResource extension for registering registry services and description of a central list 
of said IVOA registry services. (c) A Registry of Registries as the root component of data discovery 
in the VO. 

\begin{figure}[h]
\centering
\includegraphics[width=0.70\textwidth]{ivoa-arch-registry.pdf}
\caption{Registry Standards and Dependencies}
\label{fig:regdeps}
\end{figure}

There are a number of standards associated with the registry to enable registration and discovery 
of services in the Registry.  Figure \ref{fig:regdeps} shows how various Registry standards relate to each other.

\subsection{Identifier}

An IVOA Resource Identifier (or IVOA identifier or IVOA ID for short) \citep{2016ivoa.spec.0523D} is a globally unique reference 
to a resource represented in a compact, ASCII-text format.  An IVOA identifier MUST always refer to 
a resource that has been registered with an IVOA-compliant registry; that is, it should be possible 
to use the ID to get a description of the resource from a compliant registry somewhere in the VO 
environment.

\subsection{VOResource}

VOResource \citep{2018ivoa.spec.0625P} describes an encoding standard for IVOA Resource Metadata. The primary intended use of 
VOResource is to provide an XML interchange format for use with resource registries. A registry is 
a repository of resource descriptions and is employed by users and applications to discover resources. 
VOResource can be used to pass descriptions from publishers to registries and then from registries to 
applications. Another intended use is as a language for services to describe themselves directly.

\subsection{VODataService}

The VODataService \citep{2010ivoa.spec.1202P} standard makes discovery possible.  It is an encoding standard that enables one to 
describe how the data underlying the resource covers the sky as well as their frequency and time. 
VODataService also enables detailed descriptions of tables that include information useful to the 
discovery of tabular data. 

\subsection{Registry Interface}

The Registry Interface \citep{2018ivoa.spec.0723D} defines the interfaces that support interactions between applications and 
registries as well as between the registries themselves.  It is based on a general, distributed 
model composed of searchable and publishing registries. The specification has two main components: 
(a) an interface for searching and (b) an interface for harvesting.  Finally, Registry Interface 
details the metadata used to describe registries themselves as resources using an extension of the 
VOResource metadata schema.

\subsection{Resource Metadata}

The Resource Metadata \citep{2007ivoa.spec.0302H} standard represents the essential capability to describe what data and 
computational facilities are available where, and once identified, how to use them.  The data 
themselves have associated metadata (e.g., FITS keywords), and similarly we require metadata 
about data collections and data services so that VO users can easily find information of interest.  

\subsection{RegTAP}

The Registry Relational Schema for Table Access Protocol (RegTAP) \citep{2019ivoa.spec.1011D} provides a mechanism 
with which VO applications can discover and select resources - first and foremost data and services 
- that are relevant for a particular scientific problem. This specification defines an interface 
for searching this resource metadata based on the IVOA's TAP protocol. It specifies a set of tables 
that comprise a useful subset of the information contained in the registry records, as well as the 
table's data content in terms of the XML VOResource data model. The general design of the system 
is geared towards allowing easy authoring of queries. 

\subsection{SimpleDALRegExt}

Describing Simple Data Access Services (SimpleDALRegExt) \citep{2017ivoa.spec.0530P} 
is part of the registry standards that make discovery of Simple DAL services 
possible (e.g., SIAP, SCS, SSAP, SLAP).  SimpleDALRegExt refers to an encoding standard for a 
specialized extension of the IVOA Resource Metadata that is useful for describing VO Simple DAL 
Services.  By registering a VO Application in a Registry, it gets a unique IVOA Resource 
Identifier which then can be referred to by other applications and services. 

\subsection{StandardsRegExt}

The Standards registry extension (StandardsRegExt) \citep{2012ivoa.spec.0508H} is part of the registry standards that make discovery of VO Standards possible.  
StandardsRegExt refers to an encoding standard for a specialized extension of the IVOA Resource 
Metadata that is useful for describing a VO Standard.  By registering an IVOA Standard in a 
Registry, it gets a unique IVOA Resource Identifier which then can be referred to in other 
resource descriptions, namely for services that support the standard.

\subsection{TAPRegExt}

The Table Access Protocol registry extension (TAPRegExt) \citep{2012ivoa.spec.0827D} is part of the registry standards that make discovery of VO TAP Services possible.  
TAPRegExt refers to an encoding standard for a specialized extension of the IVOA Resource Metadata 
that is useful for describing VO Applications.  By registering a VO TAP Service in a Registry, it 
gets a unique IVOA Resource Identifier which then can be referred to by other applications and 
services.  In the context of registering TAP services, an important role filled by TAPRegExt is 
the communication of supported data models to the registry.

\section{Data Model Standards}

The key element for achieving interoperability among actors sharing data is the definition 
of shared standard data models. Shared data models enable rich and robust information sharing 
between heterogeneous providers and users through a standard structure, semantics, and formats; 
data models are the foundation for this exchange.

\begin{figure}
\centering
\includegraphics[width=0.62\textwidth]{ivoa-arch-dm.pdf}
\caption{Data Model Standards and Dependencies}
\label{fig:dmdeps}
\end{figure}

\subsection{VO-DML}

The VO Data Modelling Language (VO-DML) \citep{2018ivoa.spec.0910L} defines a standard modelling language, or meta-model, for 
expressing data models in the IVOA. Adopting such a uniform language for all models allows 
these to be used in a homogeneous manner and allows a consistent definition of reuse of one 
model by another. The particular language defined here includes a consistent identification 
mechanism for model which allows these to be referenced in an explicit and uniform manner 
also from other contexts, in particular from othe IVOA standard formats such as VOTable. 
The language defined in this specification is named VO-DML (VO Data Modeling Language). 
VO-DML is a conceptual modeling language that is agnostic of serializations, or physical 
representations. This allows it to be designed to fit as many purposes as possible. VO-DML 
is directly based on UML, and can be seen as a particular representation of a UML2 Profile. 
VO-DML is restricted to describing static data structures and from UML it only uses a subset 
of the elements defined in its language for describing ``Class Diagrams''. Its concepts can 
be easily mapped to equivalent data modelling concepts in other representations such as 
relational databases, XML schemas and object-oriented computer languages. VO-DML has a 
representation as a simple XML dialect named VO-DML/XML that must be used to provide the 
formal representation of a VO-DML data model. VO-DML/XML aims to be concise, explicit and 
easy to parse and use in code that needs to interpret annotated data sets. VO-DML as 
described in this document is an example of a domain specific modeling language, where the 
domain here is defined as the set of data and meta-data structures handled in the IVOA and 
Astronomy at large. VO-DML provides a custom representation of such a language and as a 
side effect allows the creation and use of standards compliant data models outside of the 
IVOA standards context. 

\subsection{CharDM} 

The Characterisation Data Model (CharDM) \citep{2008ivoa.spec.0325L} defines the high level metadata necessary to describe the 
physical parameter space of observed or simulated astronomical data sets, such as 2D-images, 
data cubes, X-ray event lists, and IFU data. This model is an abstraction which can be used 
to derive a structured description of any relevant data and thus to facilitate its discovery 
and scientific interpretation. The model aims at facilitating the manipulation of heterogeneous 
data in any VO framework or portal. A VO Characterisation instance can include descriptions of 
the data axes, the range of coordinates covered by the data, and details of the data sampling 
and resolution on each axis. These descriptions should be in terms of physical variables, 
independent of instrumental signatures as far as possible.

\subsection{ObsCoreDM}

The Observation Data Model Core Components (ObsCoreDM) \citep{2017ivoa.spec.0509L} specifies the metadata that are necessary to perform data 
discovery when querying data centers for astronomical observations of interest. It exposes 
use-cases to be carried out, explains the model and provides guidelines for its implementation 
as a data access service based on the Table Access Protocol (TAP). It aims at providing a 
simple model easy to understand and to implement by data providers that wish to publish their 
data into the Virtual Observatory. This interface integrates data modeling and data access 
aspects in a single service and is named ObsTAP. It will be referenced as such in the IVOA 
registries. In this document, the Observation Data Model Core Components (ObsCoreDM) defines 
the core components of queryable metadata required for global discovery of observational data.
It is meant to allow a single query to be posed to TAP services at multiple sites to perform 
global data discovery without having to understand the details of the services present at each 
site. It defines a minimal set of basic metadata and thus allows for a reasonable cost of 
implementation by data providers. The combination of the ObsCoreDM with TAP is referred to 
as an ObsTAP service. As with most of the VO Data Models, ObsCoreDM makes use of STC, Utypes, 
Units and UCDs. The ObsCoreDM can be serialized as a VOTable. ObsCoreDM can make reference 
to more complete data models such as Characterisation DM, Spectrum DM or Simple Spectral Line 
Data Model (SSLDM). The current specification on the contrary provides guidelines for 
implementing these concepts using the TAP protocol and answering ADQL queries. It is 
dedicated to global discovery.

\subsection{PhotDM}

The Photometry Data Model (PhotDM) \citep{2013ivoa.spec.1005S} describes photometry filters, photometric systems, magnitude 
systems, zero points and its interrelation with the other IVOA data models through a 
simple data model. Particular attention is given necessarily to optical photometry where 
specifications of magnitude systems and photometric zero points are required to convert 
photometric measurements into physical flux density units.

\subsection{ProvenanceDM}

The Provenance Data Model \citep{2020ivoa.spec.0411S} describes how provenance information can be modeled, stored 
and exchanged within the astronomical community in a standardized way. We follow the 
definition of provenance as proposed by the W3C, i.e. that ``provenance is information 
about entities, activities, and people involved in producing a piece of data or thing, 
which can be used to form assessments about its quality, reliability or trustworthiness''.
Such provenance information in astronomy is important to enable any scientist to trace 
back the origin of a dataset (e.g. an image, spectrum, catalog or single points in a 
spectral energy distribution diagram or a light curve), a document (e.g. an article, a 
technical note) or a device (e.g. a camera, a telescope), learn about the people and 
organizations involved in a project and assess the reliability, quality as well as the 
usefulness of the dataset, document or device for her own scientific work. 

\subsection{SimDM}

The Simulation Data Model (SimDM) \citep{2012ivoa.spec.0503L} describes numerical computer simulations of astrophysical systems. 
The primary goal of this standard is to support discovery of simulations by describing those 
aspects of them that scientists might wish to query on, i.e. it is a model for meta-data 
describing simulations. This document does not propose a protocol for using this model. 
IVOA protocols are being developed and are supposed to use the model, either in its original 
form or in a form derived from the model proposed here, but more suited to the particular protocol. 

\subsection{SSLDM}

The Simple Spectral Lines Data Model (SSLDM) \citep{2010ivoa.spec.1209O} describes spectral line transitions. The main objective of 
the model is to integrate with and support the Simple Line Access Protocol, with which it forms 
a compact unit. This integration allows seamless access to Spectral Line Transitions available 
worldwide in the VO context. This model does not provide a complete description of Atomic and 
Molecular Physics, which scope is outside of this document. In the astrophysical sense, a line 
is considered as the result of a transition between two energy levels. Under the basis of this 
assumption, a whole set of objects and attributes have been derived to define properly the 
necessary information to describe lines appearing in astrophysical contexts.

\subsection{SpectralDM}

The Spectral Data Model \citep{2007ivoa.spec.1029M} describes the structure of spectrophotometric datasets with spectral and 
temporal coordinates and associated metadata. This data model may be used to represent spectra, 
time series data, segments of SED (Spectral Energy Distributions) and other spectral or temporal 
associations. 

\subsection{VOEvent} 

The VOEvent model \citep{2006ivoa.spec.1101S} defines the content and meaning of a standard information packet for 
representing, transmitting, publishing and archiving information about a transient celestial 
event, with the implication that timely follow-up is of interest. The objective is to motivate 
the observation of targets-of-opportunity, to drive robotic telescopes, to trigger archive 
searches, and to alert the community. VOEvent is focused on the reporting of photon events, 
but events mediated by disparate phenomena such as neutrinos, gravitational waves, and solar 
or atmospheric particle bursts may also be reported.

Structured data is used, rather than natural language, so that automated systems can effectively 
interpret VOEvent packets. Each packet may contain zero or more of the ``who, what, where, when, how''
of a detected event, but in addition, may contain a hypothesis (a ``why'') regarding the nature of 
the underlying physical cause of the event. Citations to previous VOEvents may be used to place 
each event in its correct context. Proper curation is encouraged throughout each event's life 
cycle from discovery through successive follow-ups. VOEvent packets gain persistent identifiers 
and are typically stored in databases reached via registries. VOEvent packets may therefore 
reference other packets in various ways. Packets are encouraged to be small and to be processed 
quickly. This standard does not define a transport layer or the design of clients, repositories, 
publishers or brokers; it does not cover policy issues such as who can publish, who can build a 
registry of events, who can subscribe to a particular registry, nor the intellectual property issues. 

\subsection{STC}

The Space-Time Coordinate (STC) \citep{2007ivoa.spec.1030R} metadata for the Virtual Observatory describes the coordinate 
axes of astronomical data. It details the various components, highlights some implementation 
considerations, presents a complete set of UML diagrams, and discusses the relation between 
STC and certain other parts of the Data Model. Two serializations are discussed: XML (STC-X) and
ascii string (STC-S); the former is an integral part of the model.

\subsection{Coords}

The Coordinates Data Model (PR) covers the following concepts: description of single and multi-dimensional 
coordinate space and coordinates within that space, cordinate frames providing metadata describing the 
origin and orientation of the coordinate space, the definition of simple domain-specific coordinate
types for the most common use cases, and description of the coordinate systems domain space. This
model is a refactored subset of the original STC data model.

\subsection{Meas}

The Measurements Data Model (PR) covers the description of measured or determined astronomical data 
to enable the association of the determined ``value'' with corresponding errors. In this model, 
the ``value'' is given by the various coordinate types of the coordinates data model plus a 
description of the error model.  This model is a refactored subset of the original STC data model.

\subsection{Transform - Draft}

The Transform Data Model (WD) covers the World Coordinate System transform component and includes 
the following concepts: the description of mathematical operations which form the building 
blocks for conversions from one coordinate space to another, and the combination of individual 
operations into an arbitrarily complex transform.
 
\subsection{DatasetDM - Draft}

The Dataset Data Model (WD) provides a data model describing the structure and content of generic 
Dataset metadata for the IVOA. This is a high-level model which is to be referenced and 
extended by other models describing specific types of Datasets and Data products. In 
this document, we specify the generic Dataset, as well as an ObservationDataset model 
which covers the class of Datasets which are derived from an Observation. At the time of 
this writing, there is no formal Observation-Experiment model for the IVOA, so we include 
a hypothetical Observation-Experiment model to serve as a placeholder. 

\subsection{CubeDM - Draft}

This Cube Data Model (WD) presents an abstracted representation of N-Dimensional cube datasets and 
serves as a framework on which to construct models for more specialized Astronomical datasets. 

\subsection{ObsLocTAP} 

The Observation Locator Table Access Protocol (ObsLocTAP, PR) defines a data model for scheduled observations 
and a method to run queries over compliant data, using several Virtual Observatory technologies.
The data model builds on the ObsCore data model, removing elements associated with dataset 
access that are not available during the planning phase. In this way, this standard is focused 
on access to metadata related to the planning of a certain observatory, more than on access to 
the scientific data products. Also, the data model will be focused on discovery of planned 
observations, which is very useful information for multi-wavelength coordination observations, 
re-planning information propagation, follow-up of Targets of Opportunity alerts, preparation 
of proposals, etc. As with ObsCore, a serialisation into a relational table is defined, which 
allows users to run complex queries using the IVOA Table Access Protocol. The document also 
prescribes how to register and discover ObsLocTAP services. 

\subsection{EPN-TAP}

The Euro Planetary Networt TAP (EPN-TAP, PR) framework describes use of TAP with the EPNcore metadata dictionary. The EPNcore 
metadata dictionary defines the core components that are necessary to perform data discovery 
in the Solar System related science fields. It includes keywords to describe data products 
coverage (temporal, spectral, spatial, photometric), origin (instrument, facility), content 
(target, physical parameters), access, references, etc. Its implementation with TAP (Table 
Access Protocol) is presented, including service registration guidelines. Topical extension 
metadata dictionaries are also presented. 

\section{Data Access Standards}
\label{dal}

The data access standards define API for querying and accessing data holdings.
These standards are primarily implemented by data providers so that the community 
can use agreed and shared tools to interact with the data holdings.

As it is visible from Fig.~\ref{fig:daldeps} interconnection of data
access standards is, currently, quite complicated, even without taking
into account general VO landscape dependencies. This depends on two main
factors: standards not yet updated to rely on DALI (Sec.~\ref{dal:dali})
and \textit{simple access} (parametric query solutions) with respect to
\textit{relational tableset based} (supported through TAP, Sec.~\ref{dal:tap}) 
protocols.

Besides that, some specific cases and standards complete or support the
data access solutions:
\begin{itemize}
	\item ADQL: a SQL-based language to bring astrophysics specific
solutions in querying relational databases;
	\item VTP: a specific transport protocol to broadcast VOEvent
messages;
	\item SimDAL: a dedicated access protocol, using SimDM structure and
concepts to allow access to simulated data collections.
\end{itemize}

\begin{figure}[h]
\centering
\includegraphics[width=0.90\textwidth]{ivoa-arch-dal.pdf}
\caption{Data Access Standards and Dependencies}
\label{fig:daldeps}
\end{figure}

Here follow brief descriptions of the access layer's standards, roughly
order as: baseline standards, datasets/records discovery, data access
solutions, peculiar standards.

\subsection{ADQL}

The Astronomical Data Query Language (ADQL) \citep{2008ivoa.spec.1030O} has been developed based on SQL92. This document 
describes the subset of the SQL grammar supported by ADQL. Special restrictions and 
extensions to SQL92 have been defined in order to support generic and astronomy 
specific operations. 

\subsection{ConeSearch}

The (simple) Cone Search \citep{2008ivoa.specQ0222P} API specification defines a simple query protocol for retrieving 
records from a catalog of astronomical sources. The query describes sky position and an 
angular distance, defining a cone on the sky. The response returns a list of astronomical 
sources from the catalog whose positions lie within the cone, formatted as a VOTable. 

\subsection{DALI}
\label{dal:dali}

The Data Access Layer Interface (DALI) \citep{2017ivoa.spec.0517D} defines the base web service interfaces common to all Data 
Access Layer (DAL) services. This standard defines the behaviour of common resources, the 
meaning and use of common parameters, success and error responses, and DAL service 
registration. The goal of this specification is to define the common elements that are 
shared across DAL services in order to foster consistency across concrete DAL service 
specifications and to enable standard re-usable client and service implementations and 
libraries to be written and widely adopted. 

\subsection{DataLink}

The DataLink \citep{2015ivoa.spec.0617D} API specification describes the linking of data discovery metadata to access to 
the data itself, further detailed metadata, related resources, and to services that perform 
operations on the data. The web service capability supports a drill-down into the details 
of a specific dataset and provides a set of links to the dataset file(s) and related resources. 
This specification also includes a VOTable-specific method of providing descriptions of one 
or more services and their input(s), usually using parameter values from elsewhere in the 
VOTable document. Providers are able to describe services that are relevant to the records 
(usually datasets with identifiers) by including service descriptors in a result document. 

\subsection{ObjVisSAP - Draft}

The Object Visibility Simple Access Protocol (ObjVisSAP, WD) is an IVOA Data Access protocol 
which defines the standard for retrieving object constraint-free visibility time intervals 
through a uniform interface within the VO framework for given object coordinates to be 
observed by a given Astronomical Observatory. The ObjVisSAP interface is meant to be 
reasonably simple to be implemented by service providers. A basic query will be done 
introducing a set of sky coordinates and a given time period (optional). The service 
returns a list of constraint-free visibility time intervals formatted as VOTable. Thus, 
an implementation of the service may support additional search parameters (some of which 
may be custom to that particular service) to more finely control the selection of the 
visibility periods. The specification also describes how the search on extra parameters 
has to be done.

\subsection{SIA}

The Simple Image Access (SIA) \citep{2015ivoa.spec.1223D} protocol provides capabilities for the discovery, description, 
access, and retrieval of multi-dimensional image datasets, including 2-D images as well 
as datacubes of three or more dimensions. SIA data discovery is based on the ObsCore Data 
Model, which primarily describes data products by the physical axes (spatial, spectral, 
time, and polarization). Image datasets with dimension greater than 2 are often referred 
to as datacubes, cube or image cube datasets and may be considered examples of hypercube 
or n-cube data. In this document the term ``image'' refers to general multi-dimensional 
datasets and is synonymous with these other terms unless the image dimensionality is 
otherwise specified. SIA provides capabilities for image discovery and access. Data 
discovery and metadata access (using ObsCoreDM) are defined here. The capabilities for 
drilling down to data files (and related resources) and services for remote access are 
defined elsewhere, but SIA also allows for direct access to retrieval. 

\subsection{SimDAL}

The Simulation Data Access Layer (SimDAL) \citep{2017ivoa.spec.0320L} protocol defines a set of resources and associated 
actions to discover and retrieve simulations and numerical models in the Virtual Observatory. 
SimDAL and the Simulation Data Model are dedicated to cover the needs for the publication 
and retrieval of any kind of simulations: N-body or MHD simulations, numerical models of 
astrophysical objects and processes, theoretical synthetic spectra, etc... SimDAL is 
divided in three parts. First, SimDAL Repositories store the descriptions of theoretical 
projects and numerical codes. They can be used by clients to discover theoretical 
services associated with projects of interest. Second, SimDAL Search services are 
dedicated to the discovery of precise datasets. Finally, SimDAL Data Access services 
are dedicated to retrieve the original simulation output data, as plain raw data or 
formatted datasets cut-outs. To manage any kind of data, eventually large or at 
high-dimensionality, the SimDAL standard lets publishers choose any underlying 
implementation technology. 

\subsection{SLAP}

The Simple Line Access Protocol (SLAP) \citep{2010ivoa.specQ1209O} is an IVOA data access protocol which defines a protocol 
for retrieving spectral lines coming from various Spectral Line Data Collections through a 
uniform interface within the VO framework. These lines can be either observed or theoretical 
and will be typically used to identify emission or absorption features in astronomical 
spectra. It makes use of the Simple Spectral Line Data Model to characterize spectral lines 
through the use of utypes. The SLAP interface is meant to be reasonably simple to implement 
by service providers. A basic query will be done in a wavelength range for the different 
services. The service returns a list of spectral lines formatted as a VOTable. Thus, an 
implementation of the service may support additional search parameters (some which may be 
custom to that particular service) to more finely control the selection of spectral lines.
The specification also describes how the search on extra parameters has to be done, making 
use of the support provided by the Simple Spectral Line Data Model

\subsection{SSAP}

The Simple Spectral Access Protocol (SSAP) \citep{2012ivoa.spec.0210T} defines a uniform interface to remotely discover and 
access one dimensional spectra. SSA is a member of an integrated family of data access 
altogether comprising the Data Access Layer (DAL) of the IVOA. SSA is based on a more 
general data model capable of describing most tabular spectrophotometric data, including 
time series and spectral energy distributions (SEDs) as well as 1-D spectra; however the 
scope of the SSA interface as specified in this document is limited to simple 1-D spectra,
including simple aggregations of 1-D spectra. The form of the SSA interface is simple: 
clients first query the global resource registry to find services of interest and then 
issue a data discovery query to selected services to determine what relevant data is 
available from each service; the candidate datasets available are described uniformly 
in a VOTable format document which is returned in response to the query. Finally, the 
client may retrieve selected datasets for analysis. Spectrum datasets returned by an SSA 
spectrum service may be either precomputed, archival datasets, or they may be virtual 
data which is computed on the fly to respond to a client request. Spectrum datasets may 
conform to a standard data model defined by SSA, or may be native spectra with custom 
project-defined content. Spectra may be returned in any of a number of standard data 
formats. Spectral data is generally stored externally to the VO in a format specific to 
each spectral data collection; currently there is no standard way to represent astronomical 
spectra, and virtually every project does it differently. Hence spectra may be actively 
mediated to the standard SSA-defined data model at access time by the service, so that 
client analysis programs do not have to be familiar with the idiosyncratic details of each 
data collection to be accessed. 

\subsection{SODA} 

The Server-side Operations for Data Access (SODA) \citep{2017ivoa.spec.0517B} API for low-level data access or server side 
data processing. The initial version describes operations for extracting a subsection of a data
file using astronomical coordinates; Future evolution is expected to include performing 
various kinds of operations: transformations, pixel operations, and applying functions to the data.

\subsection{TAP}
\label{dal:tap}

The Table Access Protocol (TAP) \citep{2019ivoa.spec.0927D} defines a service protocol for accessing general table data, 
including astronomical catalogs as well as general database tables. Access is provided 
for both database and table metadata as well as for actual table data. This version of 
the protocol includes support for multiple query languages, including queries specified 
using the Astronomical Data Query Language within an integrated interface. It also 
includes support for both synchronous and asynchronous queries. Special support is 
provided for spatially indexed queries using the spatial extensions in ADQL. A multi-position 
query capability permits queries against an arbitrarily large list of astronomical targets, 
providing a simple spatial cross-matching capability. More sophisticated distributed 
cross-matching capabilities are possible by orchestrating a distributed query across 
multiple TAP services. 

\subsection{VTP}

The VOEvent Transport Protocol (VTP) \citep{2017ivoa.spec.0320S} formalizes a TCP-based protocol for VOEvent transportation 
that has been in use by members of the VOEvent community for several years and discusses 
the topology of the event distribution network. It is intended to act as a reference for 
the production of compliant protocol implementations. 

\section{Infrastructure Resource Standards}
\label{gws}

Infrastructure resource standards define or sanction APIs and formats to support for access 
to shared resources: computing, storage, and science platforms. These standards borrow from
or sanction industry standards or provide a common abstraction for users that can be implemented 
on top of industry standard infrastructure.

\begin{figure}[h]
\centering
\includegraphics[width=0.54\textwidth]{ivoa-arch-gws.pdf}
\caption{Authentication and Authorization Standards and Dependencies}
\label{fig:gwsdeps}
\end{figure}

\subsection{PDL}

The Parameter Description Language (PDL) \citep{2014ivoa.spec.0523Z} defines a language where parameters are described in a 
rigorous data model. With no loss of generality, we will represent this data model using 
XML. It intends to be a expressive language for self-descriptive web services exposing 
the semantic nature of input and output parameters, as well as all necessary complex 
constraints. PDL is a step forward towards true web services interoperability. 
 
\subsection{UWS} 

The Universal Worker Service (UWS) \citep{2016ivoa.spec.1024H} pattern defines how to manage asynchronous execution 
of jobs on a service. Any application of the pattern defines a family of related services 
with a common service contract. Possible uses of the pattern are also described. 

\subsection{VOSI} 

This VO Service Interface (VOSI) \citep{2017ivoa.spec.0524G} describes the minimum interface that a web service requires to 
participate in the world-wide network of VO services. Note that this is not required of 
standard VO services developed prior to this specification, although uptake is strongly 
encouraged on any subsequent revision. All new standard VO services, however, must feature 
a VOSI-compliant interface. 

\subsection{VOSpace}

The VOSpace \citep{2018ivoa.spec.0621G} API defines an interface to distributed storage. This specification presents the 
second RESTful version of the interface. It specifies how VO agents and applications can 
use network attached data stores to persist and exchange data in a standard way. 

\bibliography{ivoatex/ivoabib,ivoatex/docrepo,localrefs}

\end{document}
